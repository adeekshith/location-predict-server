

\documentclass[conference]{IEEEtran}
\usepackage{graphicx}
\usepackage{url}
\usepackage{listings}

% *** GRAPHICS RELATED PACKAGES ***
%
\ifCLASSINFOpdf
  % \usepackage[pdftex]{graphicx}
  % declare the path(s) where your graphic files are
  % \graphicspath{{../pdf/}{../jpeg/}}
  % and their extensions so you won't have to specify these with
  % every instance of \includegraphics
  % \DeclareGraphicsExtensions{.pdf,.jpeg,.png}
\else
  % or other class option (dvipsone, dvipdf, if not using dvips). graphicx
  % will default to the driver specified in the system graphics.cfg if no
  % driver is specified.
  % \usepackage[dvips]{graphicx}
  % declare the path(s) where your graphic files are
  % \graphicspath{{../eps/}}
  % and their extensions so you won't have to specify these with
  % every instance of \includegraphics
  % \DeclareGraphicsExtensions{.eps}
\fi



\begin{document}
%
% paper title
% can use linebreaks \\ within to get better formatting as desired
\title{Location Prediction Based Action}


% author names and affiliations
% use a multiple column layout for up to three different
% affiliations
\author{\IEEEauthorblockN{Deekshith Allamaneni}
\IEEEauthorblockA{Dept. of Electrical \& Computer Engineering\\
Missouri University of Science and Technology, Rolla, Missouri 65409\\
Email: daqnf@mst.edu}
}

% make the title area
\maketitle


\begin{abstract}
%\boldmath
The aim of this project is to  predict the current location of users based on their past location history and take an action depending on the location match. In this project, the GPS sensor in the mobile phones is used to track the current location of the user and the data is logged onto a server at a defined interval. The server processes location coordinates logged over time using a neural network and estimates the current location of the user. The client requests the predicted location from server and takes an action like notifying or alerting the user if the user is not found to be in an incorrent location at that time.
\end{abstract}

% creates the second title. It will be ignored for other modes.
\IEEEpeerreviewmaketitle



\section{Introduction}
% no \IEEEPARstart
% You must have at least 2 lines in the paragraph with the drop letter
% (should never be an issue)
Many a times we miss a schedule due to forgetfulness. Reminders and alarms solve this issue to some extent but there can be times when we even forget to set an alarm. This project aims to solve this problem by keeping track of users' location and then generates a prediction of their current location by using neural networks. The current location and the predicted location is compared and necessary action is taken by the client side application.

This projects consists of two components, the client which is a mobile application and a server which processes and returns the data sent from the client. The main scope of this application is to predict the user's current location. The methods and functions to get an estimation as well as for location comparison are built into the client application and are demostrated. However, the applications of this can be extended with time using the existing client and server infrastructure.

% _____________________________________________________________________________
% CLIENT SIDE APPLICATION
\section{Client Application}
Most smart phones are equipped with a GPS sensor and also provide an API for applications to query its current geographical location. For this project I have designed a mobile application in Java for Android operating system that acquires the current location of the user by using the GPS sensor equipped with the smartphone.

The client has two main functions.
\begin{enumerate}
  \item Data Logging
  \item Location prediction based action
\end{enumerate}

\subsection{Data Logging}
The client application acquires the current location of the user via Android API and uses the REST API provided by the server side application to send the location and the time at which the location is acquired to the server.

\begin{table}[h!]
  \centering
  \resizebox{0.5\textwidth}{!}{  
  \begin{tabular}{|c|c|c|c|c|c|}
  \hline
  & Latitude & Longitude & Weekday & Hours & Minutes \\

  \hline
  \hline
  Range & -90 to +90 & -180 to +180 & 0 to 6 & 0 to 23 & 0 to 59 \\
  \hline  
\end{tabular}
}
  \caption{Data stored to server by the client}
  \label{tab:label_client_data}
\end{table}

The Table \ref{tab:label_client_data} shows the data that the clinet saves periodically to the server.

\subsection{Location prediction based action}
The server also provides a HTTP REST API to query the predicted location of the user. The client requests the current predicted location of the user using HTTP GET request. The client also gets the current GPS location from the sensor and takes an action depending on the comparison.

\subsubsection{Action 1: Report known location but prediction not matched}
When the user is at a known location but not at the right time, it notifies the user as shown in Figure \ref{fig:clinet_prediction_mismatch} and provides an interface for user acknowledgement.

\begin{figure}[h]
  \centering
  \includegraphics[width=0.5\textwidth]{images/prediction-mismatch-clinet-scr.png}
  \caption[Prediction Mismatch]
   {Screenshot of the client app displaying prediction mismatch notification}
   \label{fig:clinet_prediction_mismatch}
\end{figure}

\subsubsection{Action 2: Unidentified location}

\begin{figure}[h]
  \centering
  \includegraphics[width=0.5\textwidth]{images/unknown-location-clinet.png}
  \caption[Notifying unknown location]
   {Screenshot of the client app notifying about unknown location}
   \label{fig:clinet_notify_unknown_location}
\end{figure}

\begin{figure}[h]
  \centering
  \includegraphics[width=0.5\textwidth]{images/enter-location-name.png}
  \caption[Request Location Identity]
   {Screenshot of the client app interface to enter location name}
   \label{fig:clinet_enter_location}
\end{figure}

When the user is at an unknown location for a certain amount of time, the client side application requests the user to enter the name of this current location. If no name is provided but if the prediction reports that location, it just identifies that location by the coordinates. Figure \ref{fig:clinet_notify_unknown_location} shows the unidentified location notification and Figure \ref{fig:clinet_enter_location} shows the interface to enter a new location.

% _____________________________________________________________________________
% SERVER SIDE APPLICATION

\section{Server Application}
Server side application is written in Python using Flask web framework. Most of the heavy load is handled by the server as the client runs on a mobile application, so doing major processing on the server side can save the battery life on client side.

The server application provides a RESTful API for the clients to interact with it. It saves the data sent by the server in  a SQLite database and processes it to estimate the current location of the user.

\subsection{Neural Network}
The server application uses the Levenberg-Marquardt backpropagation neural networks to process the data sent by the client to predict the user location and returns to the client upon request.

We are using a supervised training model in which the input is the time information and the output is the location information at that corresponding time.


\begin{thebibliography}{1}

%\bibitem{IEEEhowto:kopka}
%H.~Kopka and P.~W. Daly, \emph{A Guide to \LaTeX}, 3rd~ed.\hskip 1em plus
%  0.5em minus 0.4em\relax Harlow, England: Addison-Wesley, 1999.

\bibitem{amqp-restfulapi}
Joel L. Fernandes and Ivo C. Lopes, \emph{Performance Evaluation of RESTful Web Services and AMQP Protocol}, Ubiquitous and Future Networks (ICUFN), Da nang.\hskip 1em plus
  0.5em minus 0.4em\relax IEEE 2013 Fifth International Conference on, July 2013.

\bibitem{l2}
Zanella, A. ; Bui, N. ; Castellani, A. ; Vangelista, L. ; Zorzi, M, Internet of Things for Smart Cities, Internet of Things Journal, IEEE 2014, Volume: 1 , Issue: 1, Page(s): 22 - 32. 

\bibitem{l3}
Sye Loong Keoh ; Kumar, S.S. ; Tschofenig, H, Securing the Internet of Things: A Standardization Perspective, Internet of Things Journal, IEEE 2014 Volume: 1 , Issue: 3, Page(s): 265 - 275.

\end{thebibliography}




\appendices

\section{CoAP Client}
\begin{lstlisting}
import SOAPpy
import time
i=0
fo = open("coap-results.txt", "w")
while i<1000:
	start_time = time.time()
	server = SOAPpy.SOAPProxy \
	("http://localhost:8080/")
	time_taken= time.time() \
	- start_time
	fo.write(str("%.10f" % time_taken) \
	+","+str(len(server.hello()))+"\n")
	i+=1
fo.close()
print server.hello()
\end{lstlisting}

\section{CoAP Server}
\begin{lstlisting}
import SOAPpy
def hello():
    return "Hello World!"
server = SOAPpy.SOAPServer \
	(("localhost", 8080))
server.registerFunction(hello)
server.serve_forever()
\end{lstlisting}

% that's all folks
\end{document}


