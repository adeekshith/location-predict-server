

\documentclass[conference]{IEEEtran}
\usepackage{graphicx}
\usepackage{url}
\usepackage{listings}

\usepackage{color}

\definecolor{dkgreen}{rgb}{0,0.6,0}
\definecolor{gray}{rgb}{0.5,0.5,0.5}
\definecolor{mauve}{rgb}{0.58,0,0.82}

% *** GRAPHICS RELATED PACKAGES ***
%
\ifCLASSINFOpdf
  % \usepackage[pdftex]{graphicx}
  % declare the path(s) where your graphic files are
  % \graphicspath{{../pdf/}{../jpeg/}}
  % and their extensions so you won't have to specify these with
  % every instance of \includegraphics
  % \DeclareGraphicsExtensions{.pdf,.jpeg,.png}
\else
  % or other class option (dvipsone, dvipdf, if not using dvips). graphicx
  % will default to the driver specified in the system graphics.cfg if no
  % driver is specified.
  % \usepackage[dvips]{graphicx}
  % declare the path(s) where your graphic files are
  % \graphicspath{{../eps/}}
  % and their extensions so you won't have to specify these with
  % every instance of \includegraphics
  % \DeclareGraphicsExtensions{.eps}
\fi



\begin{document}
%
% paper title
% can use linebreaks \\ within to get better formatting as desired
\title{Location Prediction Based Action}


% author names and affiliations
% use a multiple column layout for up to three different
% affiliations
\author{\IEEEauthorblockN{Deekshith Allamaneni}
\IEEEauthorblockA{Dept. of Electrical \& Computer Engineering\\
Missouri University of Science and Technology, Rolla, Missouri 65409\\
Email: daqnf@mst.edu}
}

% make the title area
\maketitle


\begin{abstract}
%\boldmath
The aim of this project is to  predict the current location of users based on their past location history and take an action depending on the location match. In this project, the GPS sensor in the mobile phones is used to track the current location of the user and the data is logged onto a server at a defined interval. The server processes location coordinates logged over time using a neural network and estimates the current location of the user. The client requests the predicted location from server and takes an action like notifying or alerting the user if the user is not found to be in an incorrent location at that time.
\end{abstract}

% creates the second title. It will be ignored for other modes.
\IEEEpeerreviewmaketitle



\section{Introduction}
% no \IEEEPARstart
% You must have at least 2 lines in the paragraph with the drop letter
% (should never be an issue)
Many a times we miss a schedule due to forgetfulness. Reminders and alarms solve this issue to some extent but there can be times when we even forget to set an alarm. This project aims to solve this problem by keeping track of users' location and then generates a prediction of their current location by using neural networks. The current location and the predicted location is compared and necessary action is taken by the client side application.

This projects consists of two components, the client which is a mobile application and a server which processes and returns the data sent from the client. The main scope of this application is to predict the user's current location. The methods and functions to get an estimation as well as for location comparison are built into the client application and are demostrated. However, the applications of this can be extended with time using the existing client and server infrastructure.

% _____________________________________________________________________________
% CLIENT SIDE APPLICATION
\section{Client Application}
Most smart phones are equipped with a GPS sensor and also provide an API for applications to query its current geographical location. For this project I have designed a mobile application in Java for Android operating system that acquires the current location of the user by using the GPS sensor equipped with the smartphone.

The client has two main functions.
\begin{enumerate}
  \item Data Logging
  \item Location prediction based action
\end{enumerate}

\subsection{Data Logging}
The client application acquires the current location of the user via Android API and uses the REST API provided by the server side application to send the location and the time at which the location is acquired to the server.

\begin{table}[h!]
  \centering
  \resizebox{0.5\textwidth}{!}{  
  \begin{tabular}{|c|c|c|c|c|c|}
  \hline
  & Latitude & Longitude & Weekday & Hours & Minutes \\

  \hline
  \hline
  Range & -90 to +90 & -180 to +180 & 0 to 6 & 0 to 23 & 0 to 59 \\
  \hline  
\end{tabular}
}
  \caption{Data stored to server by the client}
  \label{tab:label_client_data}
\end{table}

The Table \ref{tab:label_client_data} shows the data that the clinet saves periodically to the server.

\subsection{Location prediction based action}
The server also provides a HTTP REST API to query the predicted location of the user. The client requests the current predicted location of the user using HTTP GET request. The client also gets the current GPS location from the sensor and takes an action depending on the comparison.

\subsubsection{Action 1: Report known location but prediction not matched}
When the user is at a known location but not at the right time, it notifies the user as shown in Figure \ref{fig:clinet_prediction_mismatch} and provides an interface for user acknowledgement.

\begin{figure}[h]
  \centering
  \includegraphics[width=0.5\textwidth]{images/prediction-mismatch-clinet-scr.png}
  \caption[Prediction Mismatch]
   {Screenshot of the client app displaying prediction mismatch notification}
   \label{fig:clinet_prediction_mismatch}
\end{figure}

\subsubsection{Action 2: Unidentified location}

\begin{figure}[h]
  \centering
  \includegraphics[width=0.5\textwidth]{images/unknown-location-clinet.png}
  \caption[Notifying unknown location]
   {Screenshot of the client app notifying about unknown location}
   \label{fig:clinet_notify_unknown_location}
\end{figure}

\begin{figure}[h]
  \centering
  \includegraphics[width=0.5\textwidth]{images/enter-location-name.png}
  \caption[Request Location Identity]
   {Screenshot of the client app interface to enter location name}
   \label{fig:clinet_enter_location}
\end{figure}

When the user is at an unknown location for a certain amount of time, the client side application requests the user to enter the name of this current location. If no name is provided but if the prediction reports that location, it just identifies that location by the coordinates. Figure \ref{fig:clinet_notify_unknown_location} shows the unidentified location notification and Figure \ref{fig:clinet_enter_location} shows the interface to enter a new location.

% _____________________________________________________________________________
% SERVER SIDE APPLICATION

\section{Server Application}
Server side application is written in Python using Flask web framework. Most of the heavy load is handled by the server as the client runs on a mobile application, so doing major processing on the server side can save the battery life on client side.

The server application provides a RESTful API for the clients to interact with it. It saves the data sent by the server in  a SQLite database and processes it to estimate the current location of the user.

\subsection{Neural Network}
The server application uses Levenberg-Marquardt backpropagation neural networks to process the data sent by the client to predict the user location and returns to the client upon request.

We are using a supervised training model in which the input is the time information and the output is the location information at that corresponding time.

\subsubsection{Neural Network Training Inputs and Their Representation}
The inputs for the neural network is the time information including the weekday, hours and minutes.
The neural network has a total of ten inputs out of which 3 inputs are used to represent the weekday, 5 inputs to represent hour and 2 inputs for quantized minutes information.
\paragraph{Weekday}
Weekday is stored in the database in the range 0 to 6 where 0 is Monday, 1 is Tuesday and so on upto 6 for Sunday.
But while giving as an input for the neural network, we are using a binary form of the weekday so that there are three neural network inputs representing it.
\paragraph{Hours}
Hours are stored in 24-hour format ranging from 0 to 23 where 0 represents 12:00 AM and 23 represents 11:00 PM.
While passing it to the neural network, I am converting it into binary format to represent the hours information with five neural network inputs.
\paragraph{Minutes}
The client sends the exact minutes information ranging from 0 to 59 but that detail is not necessary for this application. So when giving it as an input for neural network, I am quantizing it to the lower fifteen minutes and representing the range 0 to 60 as just 0 to 3 converted to binary. The minutes 0 to 14 is represented as 0 (0,0), 15 to 29 as 1 (0, 1), 30 to 44 as 2 (1, 0) and 45 to 59 as 3 (1, 1). So the minutes information is represented using 2 inputs. This form of representation improves the performance of the netowrk by a great extent.

\subsubsection{Neural Network Training Targets and their Representation}
The targets for the neural network are the latitude and longitude corresponding to the time data given at the input.

Latitude ranges from -90 to +90. The altitude data is preprocessed by using the formula 

lat\_pre[] = (lat[] – max(lat[])+90

We are basically subtracting the latitude values with the maximum value of the latitude and then adding 90 to it which makes it a positive and relative distance from the user location rather than absolute geo-coordinates. This reduces the magnitude of the coordinates and improves the performance of the network.

Longitude has a range of - 180 to +180 and it is preprocessed similar to the latitude as shown above 

lon\_pre[] = (lat[] – max(lon[])+180

\subsubsection{Neural Network Architecture}
The neural network for this project makes use of 


\begin{thebibliography}{1}

%\bibitem{IEEEhowto:kopka}
%H.~Kopka and P.~W. Daly, \emph{A Guide to \LaTeX}, 3rd~ed.\hskip 1em plus
%  0.5em minus 0.4em\relax Harlow, England: Addison-Wesley, 1999.

\bibitem{amqp-restfulapi}
Joel L. Fernandes and Ivo C. Lopes, \emph{Performance Evaluation of RESTful Web Services and AMQP Protocol}, Ubiquitous and Future Networks (ICUFN), Da nang.\hskip 1em plus
  0.5em minus 0.4em\relax IEEE 2013 Fifth International Conference on, July 2013.


\end{thebibliography}




\appendices

\lstset{frame=tb,
  language=Java,
  aboveskip=3mm,
  belowskip=3mm,
  showstringspaces=false,
  columns=flexible,
  basicstyle={\small\ttfamily},
  numbers=none,
  numberstyle=\tiny\color{gray},
  keywordstyle=\color{blue},
  commentstyle=\color{dkgreen},
  stringstyle=\color{mauve},
  breaklines=true,
  breakatwhitespace=true,
  tabsize=3
}

\section{Client/MainActivity.java}

\begin{lstlisting}
/*
This is the main interface for the Android client
 */
package com.example.locationupdater;

import android.app.Activity;
import android.content.Intent;
import android.os.Bundle;
import android.view.Menu;
import android.view.MenuItem;

public class MainActivity extends Activity {
  @Override
  protected void onCreate(
    Bundle savedInstanceState) {
    super.onCreate(savedInstanceState);
    setContentView(R.layout.activity_main);
    Intent intent = new Intent(this, 
      LocationService.class);
    //intent.putExtra("ActivityStatus", "true");
    startService(intent);
  }

  @Override
  public boolean onCreateOptionsMenu(Menu menu) {
    // Inflate the menu; this adds items to
    //  the action bar if it is present.
    getMenuInflater().inflate(R.menu.main, menu);
    return true;
  }

  @Override
  public boolean onOptionsItemSelected(
    MenuItem item) {
    // Handle action bar item clicks here. 
    // The action bar will
    // automatically handle clicks on the 
    // Home/Up button, so long
    // as you specify a parent activity 
    // in AndroidManifest.xml.
    int id = item.getItemId();
    if (id == R.id.action_settings) {
      return true;
    }
    return super.onOptionsItemSelected(item);
  }
}

\end{lstlisting}


\section{Client/LocationService.java}
\begin{lstlisting}
/**
 * Runs a background service to monitor
 * the location and update it on the
 * server periodically.
 */
package com.example.locationupdater;

import java.io.BufferedReader;
import java.io.InputStreamReader;
import java.net.URL;
import java.net.URLConnection;
import java.util.Calendar;
import java.util.TimeZone;
import java.util.UUID;

import org.json.JSONException;

import android.app.Service;
import android.content.Context;
import android.content.Intent;
import android.content.SharedPreferences;
import android.content.SharedPreferences.Editor;
import android.location.Location;
import android.location.LocationListener;
import android.location.LocationManager;
import android.os.AsyncTask;
import android.os.Bundle;
import android.os.IBinder;
import android.util.Log;

public class LocationService extends Service {
   // in Meters
  private static final long MIN_DIST = 0;
   // requireed  5 mins in Milliseconds
  private static final long MON_TIME = 300000;
  //DBController dbcontroller;
  Calendar c;
  
  private static String uniqueID = null;
  private static final String PREF_UNIQUE_ID 
    = "PREF_UNIQUE_ID";
  
  //public static final String Stub = null;
  protected LocationManager locationManager;
    // LocationListener mlocList ;
    private Context mComtext;
    Location location; // location
    
    //GPSTracker mGPS;
    
  @Override
  public void onCreate() {
    // TODO Auto-generated method stub
    super.onCreate();
    Log.i("LocationService", "onCreate");
    mComtext = this;
     // US OR CST Time zone
    TimeZone tz = TimeZone.getTimeZone("GMT-06:00");
    c = Calendar.getInstance(tz);
  }
  
  
  @Override
  public int onStartCommand(
    Intent intent, 
    int flags, 
    int startId) {
    Log.i("LocationService", "onStartCommand");
    
    //mlocList = new MyLocationListener();
        
    locationManager = 
      (LocationManager) getSystemService(
        Context.LOCATION_SERVICE);
        
        locationManager.requestLocationUpdates(
                LocationManager.GPS_PROVIDER, 
                MON_TIME, 
                MIN_DIST,
                new MyLocationListener()
        );
        
    return START_STICKY;
  }
  
  @Override
  public IBinder onBind(Intent arg0) {
    // TODO Auto-generated method stub
    return null;
  }
  
  
  private class MyLocationListener implements LocationListener {

        public void onLocationChanged(Location location) {
            int dayOfWeek = c.get(Calendar.DAY_OF_WEEK) - 1;
            Log.i("LocationService dayOfWeek ", "" + dayOfWeek);
            int hour = c.get(Calendar.HOUR_OF_DAY);
            Log.i("LocationService hour ", "" + hour);
            int minutes = c.get(Calendar.MINUTE);
            if(minutes < 15){
              minutes = 0;
            }else if(minutes < 30){
              minutes = 1;
            }else if(minutes < 45){
              minutes = 2;              
            }else{
              minutes = 3;
            }
            Log.i("LocationService minutes ", "" + minutes);
            //dbcontroller.logLocationData(location.getLatitude()
              ,  location.getLongitude(), dayOfWeek, hour, minutes);
            
            String uuid = getUUID(mComtext);
            String url = "http://parishod.com/logdata/" +
               uuid + "/" + 
               location.getLatitude() + "/" + 
               location.getLongitude() + "/" + 
               dayOfWeek + "/" + 
               hour + "/" + minutes;
           
            Log.i("LocationService URL ", "" + url);
            new MyAsyncTask().execute(url);
        }

    }
  
  private class MyAsyncTask extends AsyncTask<String, Integer, Double>{
    //String result1 = "";
    @Override
    protected Double doInBackground(String... params) {
      // TODO Auto-generated method stub
      try {
        postData(params[0]);
      } catch (JSONException e) {
        // TODO Auto-generated catch block
        e.printStackTrace();
      } catch (Exception e) {
        // TODO Auto-generated catch block
        e.printStackTrace();
      }
      return null;
    }

    public void postData(String valueIWantToSend) throws Exception {
      //String resultString = null;
      String content = getResponse(valueIWantToSend);
          System.out.println(content);
          
          String uuid = getUUID(mComtext);
          String content1 = getResponse(
            "http://parishod.com/logdata/" + 
            uuid + "/predictlocation");
          System.out.println(content1);           
    }
 
  }
  
  /*Function to get UUID*/
  public String getUUID(Context context) {
      if (uniqueID == null) {
          SharedPreferences sharedPrefs = 
            context.getSharedPreferences(
                  PREF_UNIQUE_ID, Context.MODE_PRIVATE);
          uniqueID = sharedPrefs.getString(PREF_UNIQUE_ID, null);
          if (uniqueID == null) {
              uniqueID = UUID.randomUUID().toString();
              Editor editor = sharedPrefs.edit();
              editor.putString(PREF_UNIQUE_ID, uniqueID);
              editor.commit();
          }
      }
      return uniqueID;
  }
  
  /*Gets Json Response*/
  public static String getResponse(String url) throws Exception {
        URL website = new URL(url);
        URLConnection connection = website.openConnection();
        BufferedReader in = new BufferedReader(
                                new InputStreamReader(
                                    connection.getInputStream()));

        StringBuilder response = new StringBuilder();
        String inputLine;

        while ((inputLine = in.readLine()) != null) 
            response.append(inputLine);

        in.close();

        return response.toString();
    }
  
}

\end{lstlisting}


\section{Client/GPSTracker.java}
\begin{lstlisting}
/**
 * Gets the current location information 
 * from the GPS using native android API
 */
package com.example.locationupdater;


import android.app.AlertDialog;
import android.content.Context;
import android.content.DialogInterface;
import android.content.Intent;
import android.location.Location;
import android.location.LocationListener;
import android.location.LocationManager;
import android.os.Bundle;
import android.provider.Settings;
import android.util.Log;

public final class GPSTracker implements LocationListener {

    private final Context mContext;

    // flag for GPS status
    public boolean isGPSEnabled = false;

    // flag for network status
    boolean isNetworkEnabled = false;

    // flag for GPS status
    boolean canGetLocation = false;

    Location location; // location
    double latitude; // latitude
    double longitude; // longitude

    // The minimum distance to change Updates in meters
    private static final long MIN_DIST = 1; // 10 meters

    // The minimum time between updates in milliseconds
    private static final long MIN_TIME_UPD = 1; // 1 minute

    // Declaring a Location Manager
    protected LocationManager locationManager;

    public GPSTracker(Context context) {
        this.mContext = context;
        getLocation();
    }

    /**
     * Function to get the user's current location
     * 
     * @return
     */
    public Location getLocation() {
        try {
            locationManager = (LocationManager) mContext
                    .getSystemService(
                        Context.LOCATION_SERVICE);

            // getting GPS status
            isGPSEnabled = locationManager
                    .isProviderEnabled(
                        LocationManager.GPS_PROVIDER);

            Log.v("isGPSEnabled", "=" + isGPSEnabled);

            // getting network status
            isNetworkEnabled = locationManager
                    .isProviderEnabled(
                        LocationManager.NETWORK_PROVIDER);

            Log.v("isNetworkEnabled", "=" + isNetworkEnabled);

            if (isGPSEnabled == false && 
                isNetworkEnabled == false) {
                // no network provider is enabled
            } else {
                this.canGetLocation = true;
                if (isNetworkEnabled) {
                    location=null;
                    locationManager.requestLocationUpdates(
                            LocationManager.NETWORK_PROVIDER,
                            MIN_TIME_UPD,
                            MIN_DIST, this);
                    Log.d("Network", "Network");
                    if (locationManager != null) {
                        location = locationManager
                          .getLastKnownLocation(
                            LocationManager.NETWORK_PROVIDER);
                        if (location != null) {
                            latitude = location.getLatitude();
                            longitude = location.getLongitude();
                        }
                    }
                }
                // if GPS Enabled get lat/long using GPS Services
                if (isGPSEnabled) {
                    location=null;
                    if (location == null) {
                        locationManager.requestLocationUpdates(
                                LocationManager.GPS_PROVIDER,
                                MIN_TIME_UPD,
                                MIN_DIST, this);
                        Log.d("GPS Enabled", "GPS Enabled");
                        if (locationManager != null) {
                            location = locationManager
                                    .getLastKnownLocation(
                                        LocationManager.GPS_PROVIDER);
                            if (location != null) {
                                latitude = location.getLatitude();
                                longitude = location.getLongitude();
                            }
                        }
                    }
                }
            }

        } catch (Exception e) {
            e.printStackTrace();
        }

        return location;
    }

    /**
     * Stop using GPS listener Calling this 
     * function will stop using GPS in the app
     * */
    public void stopUsingGPS() {
        if (locationManager != null) {
            locationManager.removeUpdates(
              GPSTracker.this);
        }
    }

    /**
     * Function to get latitude
     * */
    public double getLatitude() {
        if (location != null) {
            latitude = location.getLatitude();
        }

        // return latitude
        return latitude;
    }

    /**
     * Function to get longitude
     * */
    public double getLongitude() {
        if (location != null) {
            longitude = location.getLongitude();
        }

        // return longitude
        return longitude;
    }

    /**
     * Function to check GPS/wifi enabled
     * 
     * @return boolean
     * */
    public boolean canGetLocation() {
        return this.canGetLocation;
    }

    /**
     * Function to show settings alert dialog 
     * On pressing Settings button will
     * lauch Settings Options
     * */
    public void showSettingsAlert() {
        AlertDialog.Builder alertDialog = 
            new AlertDialog.Builder(mContext);

        // Setting Dialog Title
        alertDialog.setTitle("GPS is settings");

        // Setting Dialog Message
        alertDialog
                .setMessage(
                    "GPS is not enabled. 
                    Do you want to go to settings menu?");

        // On pressing Settings button
        alertDialog.setPositiveButton("Settings",
                new DialogInterface.OnClickListener() {
                    public void onClick(
                        DialogInterface dialog, int which) {
                        Intent intent = new Intent(
                          Settings.ACT_LOC_SETT);
                        mContext.startActivity(intent);
                    }
                });

        // on pressing cancel button
        alertDialog.setNegativeButton("Cancel",
                new DialogInterface.OnClickListener() {
                    public void onClick(
                        DialogInterface dialog, int which) {
                        dialog.cancel();
                    }
                });

        // Showing Alert Message
        alertDialog.show();
    }

    @Override
    public void onLocationChanged(Location location) {
    }

    @Override
    public void onProviderDisabled(String provider) {
    }

    @Override
    public void onProviderEnabled(String provider) {
    }

    @Override
    public void onStatusChanged(String provider, 
        int status, Bundle extras) {
    }

}
\end{lstlisting}


\lstset{frame=tb,
  language=Python,
  aboveskip=3mm,
  belowskip=3mm,
  showstringspaces=false,
  columns=flexible,
  basicstyle={\small\ttfamily},
  numbers=none,
  numberstyle=\tiny\color{gray},
  keywordstyle=\color{blue},
  commentstyle=\color{dkgreen},
  stringstyle=\color{mauve},
  breaklines=true,
  breakatwhitespace=true,
  tabsize=3
}

\section{Server/\_\_main\_\_.py}
\begin{lstlisting}
# API calls implemented here
from flask import Flask
from flask import render_template
import json
import os
import logdata.incoming
import logdata.predict
app = Flask(__name__)


@app.route('/')
def hello_world():
    return render_template('index.html')


# Depricated
@app.route('/logdata/<userid>/'+
    '<float:latitude>/<float:longitude>/'+
    '<int:weekday>/<int:hour>'+
    '/<int:minutesQuant>')
def logInputData(userid, latitude, 
    longitude, weekday, 
    hour, minutesQuant):
    # Log the user data
    inputData = logdata.incoming.Data(
        userid, latitude, 
        longitude, weekday, 
        hour, minutesQuant)
    responseJson = inputData.generateResponseJson()
    return str(responseJson)

# Depricated


@app.route('/logdata/<userid>/predictlocation')
def predictedLocationData(userid):
    responseJson = logdata.predict.locationPredict(userid)
    return str(responseJson)


@app.route('/location-predict/api/v1/'+
    'logdata/<userid>/<float:latitude>'+
    '/<float:longitude>/<int:weekday>/'+
    '<int:hour>/<int:minutesQuant>')
def logInputDataV1(userid, latitude, 
    longitude, weekday, 
    hour, minutesQuant):
    # Log the user data
    inputData = logdata.incoming.Data(
        userid, latitude, 
        longitude, weekday, 
        hour, minutesQuant)
    responseJson = inputData.generateResponseJson()
    return str(responseJson)


@app.route('/location-predict/api/v1/predict-res/'+
    '<userid>/<float:latitude>/<float:longitude>/'+
    '<int:weekday>/<int:hour>/<int:minutesQuant>')
def predictedLocationDataV1(userid, 
    latitude, longitude, 
    weekday, hour, 
    minutesQuant):
    responseJson = logdata.predict.locationPredict(
        userid, latitude, 
        longitude, weekday, 
        hour, minutesQuant)
    return str(responseJson)

if __name__ == '__main__':
    app.run(host='0.0.0.0')

\end{lstlisting}

\section{Server/incoming.py}
\begin{lstlisting}
#!/usr/bin/python
# -*- coding: utf-8 -*-
import json
import sqlite3


class Data:

    'Data class. Verifies input data, saves and generates output.'

    def __init__(self, uuid, latitude, longitude, weekday, hour, minuteQuantized):
        self.uuid = str(uuid)
        truncateToDigits = 6
        self.latitude = round(latitude, truncateToDigits)
        self.longitude = round(longitude, truncateToDigits)
        self.weekday = weekday
        self.hour = hour
        self.minuteQuantized = minuteQuantized
        self.inputValidity = self.validateInput()
        if self.inputValidity == 'valid':
            self.saveDataToDb()

    def validateInput(self):
        if self.latitude > 90.0 or 
            self.latitude < -90.0:
            return "Latitude {} is not \\\
            in the range between +/-180.0".format(
                self.latitude)
        elif self.longitude > 180.0 or 
            self.longitude < -180.0:
            return "Longitude {} is not \\\
                in the range between +/-180.0".format(
                    self.longitude)
        elif self.weekday > 6:
            return "Weekday {} is not \\\
                in the range (0, 6)".format(self.weekday)
        elif self.hour > 23:
            return "Hour {} is not \\\
                in the range (0, 23)".format(self.hour)
        elif self.minuteQuantized > 3:
            return "Quantized minute {} is not \\\
                in the range (0, 3)".format(self.minuteQuantized)
        else:
            return "valid"

    def generateResponseJson(self):
        responseData = {}
        responseData['error'] = {}
        if self.inputValidity == 'valid':
            responseData['error']['code'] = 0
        else:
            responseData['error']['code'] = 1
        responseData['error']['comment'] = 
            self.inputValidity
        responseJson = json.dumps(
            responseData, indent=4, sort_keys=True)
        return responseJson

    def saveDataToDb(self):
        print "Entered saveData"
        conn = sqlite3.connect('db/locationdata.db')
        print "Connected to database"
        insertSQL = 'INSERT INTO locationlog '+
            '(uuid, latitude, longitude, weekday, '+
                'hour, minute_quant, repeated_count)'+
                ' VALUES ("{}", {}, {}, {}, {}, {}, 1);'
        insertSQL = insertSQL.format(
            self.uuid, 
            self.latitude, 
            self.longitude, 
            self.weekday, 
            self.hour, 
            self.minuteQuantized)
        updateSQL = 'UPDATE locationlog SET '+
        'repeated_count= repeated_count+1 '+
        'WHERE EXISTS (SELECT * FROM locationlog '+
            'WHERE uuid = "{}" AND latitude={}'+
            ' AND longitude={} '+
            'AND weekday={} AND hour={} AND '+
            'minute_quant={});'
        updateSQL = updateSQL.format(
            self.uuid, self.latitude, 
            self.longitude, self.weekday, 
            self.hour, self.minuteQuantized)
        try:
            conn.execute(insertSQL)
        except:
            conn.execute(updateSQL)
        print "Table created successfully"
        conn.commit()
        conn.close()
        return 0

\end{lstlisting}

\section{Server/predict.py}
\begin{lstlisting}
#!/usr/bin/python
# -*- coding: utf-8 -*-
import json
import sqlite3
from pybrain.tools.shortcuts import buildNetwork
from pybrain.datasets import SupervisedDataSet
from pybrain.supervised.trainers import BackpropTrainer


def getDataFromDB(uuid, 
  weekdayCurrent, 
  hourCurrent, 
  minuteQuantCurrent):
    conn = sqlite3.connect('../db/locationdata.db')
    sqlGetUserData = 'SELECT uuid, latitude, '+
        'longitude, weekday, hour, '+
        'minute_quant, repeated_count '+
        'FROM locationlog WHERE uuid = "{}" '+
        'AND weekday = {} AND hour = {} '+
        'AND minute_quant = {};'
    locationEntryDB = conn.execute(
      sqlGetUserData.format(
        uuid, 
        weekdayCurrent, 
        hourCurrent, 
        minuteQuantCurrent) )
    locationEntryList = list(locationEntryDB)
    return locationEntryList


def generatePredictedJson(uuid, 
    predictedLat, 
    predictedLon, 
    statusSituation, 
    statusAction):
    responseData = {}
    responseData['prediction'] = {}
    responseData['prediction']['latitude'] = predictedLat
    responseData['prediction']['longitude'] = predictedLon
    responseData['status'] = {}
    responseData['status']['situation'] = 'normal'
    responseData['status']['action'] = 'none'
    responseJson = json.dumps(
      responseData, 
      indent=4, sort_keys=True)
    return responseJson


def locationPredict(uuid, latitudeCurrent, 
    longitudeCurrent, weekdayCurrent, 
    hourCurrent, minuteQuantCurrent):
    net = buildNetwork(3, 4, 2)
    ds = SupervisedDataSet(3, 2)
    userLocDataList = getDataFromDB(uuid, 
        weekdayCurrent, hourCurrent, 
        minuteQuantCurrent)
    print userLocDataList
    for userLocData in userLocDataList:
        latitudeDB = userLocData[1]
        longitudeDB = userLocData[2]
        weekdayDB = userLocData[3]
        hourDB = userLocData[4]
        minuteQuantDB = userLocData[5]
        repeatedCountDB = userLocData[6]
        for iter in range(repeatedCountDB):
            ds.addSample(
                (weekdayDB,hourDB,minuteQuantDB), 
                (latitudeDB, longitudeDB,))
    print("Dataset length: {}".format(len(ds)))
    trainer = BackpropTrainer(net, ds)
    # trainer.trainUntilConvergence()
    trainer.train()
    [predictedLat, predictedLon] = net.activate(
        [weekdayCurrent, 
        hourCurrent, minuteQuantCurrent])
    statusSituation = 'normal'
    statusAction = 'none'
    print "uuid: "
    responseJson = generatePredictedJson(
        uuid, 
        predictedLat, 
        predictedLon, 
        statusSituation, 
        statusAction)
    return responseJson

print locationPredict('apr0041',
 17.45, 78.35, 1, 11, 2)
\end{lstlisting}

% that's all folks
\end{document}


